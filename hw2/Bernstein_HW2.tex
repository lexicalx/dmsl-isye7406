\documentclass[12pt,letterpaper]{article}
\usepackage{fullpage, lastpage, enumerate, fancyhdr}
\usepackage[top=2cm, bottom=4.5cm, left=2.5cm, right=2.5cm]{geometry}
\usepackage{amsmath,amsthm,amsfonts,amssymb,amscd}
\usepackage{mathrsfs, xcolor, graphicx, subcaption, siunitx}
\usepackage{hyperref}
\graphicspath{{./images}}

\hypersetup{%
  colorlinks=true,
  linkcolor=blue,
  linkbordercolor={0 0 1}
}

\setlength{\parindent}{0.0in}
\setlength{\parskip}{0.05in}

\pagestyle{fancyplain}
\headheight 15pt
\fancyhf[FL]{Matthew Bernstein}
\fancyhf[HR]{\today}
\fancyhf[HL]{ISyE 7406 - Homework 2}
\fancyhf[FC]{}
\fancyhf[FR]{\thepage}
\headsep 1.5em

\title{ISYE 7406 - Homework 2}
\author{Matthew Bernstein}

\begin{document}
\fancypagestyle{plain}{
    \fancyhf{}
    \renewcommand{\headrulewidth}{0pt}
    \renewcommand{\footrulewidth}{0pt}
}
\maketitle
\section*{Introduction}

\section*{Exploratory Data Analysis}

\section*{Methodology}

Eight models were created from the data in order to understand the different types of Linear Regressions

\subsubsection*{Full Model}
A least squares linear model was fitted on all features, excluding \textit{siri}, \textit{density}, \textit{free} as noted above.

\subsubsection*{5 Best Feature Selection}
Two models were created by selecting the 5 best features and performing a regression against those. The first model was created by selecting the 5 features with the lowest p-scores in the full model. The second model was selected by performing an exhaustive search over all combinations of 5 features and selecting the combination with the lowest Mallow's Cp score. 

\subsection{}

\section*{Results}

\section*{Conclusion}


\end{document}